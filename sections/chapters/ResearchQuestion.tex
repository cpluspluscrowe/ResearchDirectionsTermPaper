\chapter{Research Question}


Concerning data that is composed of text and image data, there are feature interactions between data types that are measureable and in turn improve model performance?

%1. An autoencoder trained on a concatenation of each data type can learn new featurs, so that when combined with individual models the model performance will improve.  This question will demonstrate how well autoencoders work at learning multiple data types.

1. Complex data types can be trained on the same simple model (CNN and NN) so that the model outperforms either individual model CNN or NN model.  This research question will investigate how well or poorly standard models work on multiple data types. It can compare this performance with industry standards.

2. LSTMs trained on a concatenation of text and image data will outperform LSTMs from one data type.  The purpose of comparing both data types to a single data type within LSTMs is to measure how well LSTMs handle multiple data types.

3. Autoencoders trained on a concatenation of text and image data will outperform autoencoders trained on one data type.  The purpose of comparing both data types to a single data type is to measure how well autoencoders handle multiple data types.

4. Combining the output of image-based CNN and text-based NN will create a model that performs better than either individual model.

\section{Concepts}
The concepts presented in the research question explore how well each standard model performs when trained on multiple data types.  The outcome of the research will give better detail on how well each model handles multiple data types.  Moreover, the research will also include how well a model performs when feed features from each individual model.  The study will give baseline information to future researchers on how well each model handles being trained on combination of text and image data.  Future research can build on the models which perform best in both experimenting with new architectures and preprocessing the data.

\subsection{Open Research Problems in IT}

\subsubsection{1. Methodology for Autoencoders with Multiple Data Types}
Autoencoders can be used to create generative models. These work by learning ways to represent the data.  In this way, they can be used to transform data.  The related work covered a few examples of using autoencoders to transform one data type into another.  Yet, these encoders are only generative, not exact.  The models learn general patterns.  This makes autoencoders useful to learning general patterns from input to output.  Yet, there is not best way to use autoencoders with multiple data types.  There are many ways to provide data to autoencoders and arrange the orders of input and output.  This provides a gap in the field of autoencoders with multiple data types for proposing ways to best learn patterns between different data types.


\subsubsection{2. Detecting Joint Data Type Features}
There are many datasets that contain multiple data types.  Yet, there is no litmus test for identifying that correlated features exist between datasets.  The creation of a method for identifying the existence of joint data type features would aid researcher's when deciding how to approach and learn on the dataset.  The creation of a method to detect these between dataset features is an open question this paper can begin to address.

\section{Purpose Statement}

\subsection{Central Intent for the Study}

The topic of combining models of different data types offers a lot of promise.  If researcher's can find better ways to model complex data, i.e. data that contains multiple data types, then researcher's and practitioners will have better tools for analyzing real-world data.  Most data in the world is complex, interconnected, and can be represented in many ways.  For example, medical diagnosing patients makes use of visual data, e.g. X-Rays and a set of symptons, which can be represented as text data.  Google has published the `Open Images` dataset with 9 million URLs with labels and 6,000 categories, and it is likely that the images and labels are related. There are many similar datasets with images and labels, such as the 'Google-Landmarks Dataset'. NBC recently published over 200,000 Tweets concerning Russia's involvement with the U.S. elections (2018).  Tweets use images and text to convey their message, so it is likely a dataset with image and text correlating features.  Insurance companies also make use of images and text, such as traffic accident images and the accident report.  The text describes events in the image, and so provide context for the image.  The ability to connect text and image features could improve insurance reporting for accidents.  Having ways to combine the analysis of text and image data can produce models that make better predictions on these datasets. % maybe not the best place for examples

The central intent for the research is creating a methodology to improve the analysis of multi-data type datasets.  The research question poses both a methodology and architecture for analyzing this data.  The intent of the research question is to improve current model performance on these datasets, while showing that joint data type features exist.



\subsection{Is there a body of knowledge in IT you would like to contribute to?}

% Contribute to how complex models are analyzed

% How to provide this interracting data to existing models

%When it works and how to detect these relations
The body of knowledge I would like to contribute to is a better analysis of complex data types, specifically analyzing data that contains image and text data.  The task of analyzing complex data types is difficult because the complex data types interact differently, depending on their context and how they are used.  Moreover, because text and speech are different, a single machine learning model or architecture will introduce different biases on each data type, or will perform worse on one data type than the other. Many decisions must be made before a researcher can begin any analysis on text and speech data. Current practice generally analyzes each data type separately and combine their results. It is also common to transform one data type into the other for ease of processing, such as transcribing a picture into text and descriptive sentences. For example, these data types can be mixed into a single architecture that is better at analyzing the combination of data.  

I think the interests I can make research contributions to include complex data type preprocessing and both model architecture and combination of complex data types.  The research would focus on ways to build models that work on a combination of image and text data.  Many of the existing tools like attention models and the flexibility of CNN and NN models should go a long way to making a more combined analysis possible.  One challenge is the contex where the complex data analysis occurs.  Hopefully a combined model analysis and architecture will be general enough to apply to many context.  With the need of a general solution is a specific place to begin working on complex machine learning models.  One booming industry is the advertisement industry, and most ads are a combination of image and text data.  This field seems well suited to begin generating models for working with complex data types.  

\subsection{Is there an area of practice you would like to improve?}




