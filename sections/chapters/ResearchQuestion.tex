\section{Research Question}
What are the impacts on model performance in classification and regression scenarios when applying boosting with multi-type CNNs.

%When applying boosting to individual data type models, by using weakly trained multi-type CNNs for the ensemble models, how does this affect the model's performance in classification/regression?

\subsection{Expounding on the Research Question}
The idea for the research question hinges on the related work.  The related work concluded that when training multi-data type models, that these models only learn single data type features.  The implication is that joint data type features are less pronounced than single-data type features, or that because joint data type features may not match the type of the output that these features receive less negative feedback and so undergo less training.  From the reality that joint features are less pronounced, any modeling with joint features must consider single-data type features.  When this knowledge is given the context of current practice, it seems unwise to ditch single-data type models entirely.  Rather, models that detect joint features can be used in conjunction with single-data type features.  Such combination of models is known as an ensemble method, where multiple models are considered for the final output, since multiple semi-independent models tend to give better predictions than any single model.

Building on the idea of an ensemble method that uses joint data type features, the research question strong hinges on the theory that joint data type features are more complicated and nuanced than single-data type features.  The research question also acknowledges that current single-data type models are very good and have achieved good performance on existing benchmarks. It is likely that a joint model will not perform as well as either individual model, but instead aid the individual model's performance.  The general theory of aiding existing model performance sounds a lot like existing boosting and bagging methods on structured data.  These methods create weaker model, i.e. models that do not perform as well or only slightly better than random chance, and use those models to strengthen existing models.  The strengthening is done by identifying weak points in the model, e.g. misclassifications or poor regression predictions.  Boosting methods then build the weak models based on the poor performance of the main model.  The result is a main model with multiple weaker models to that offset weaknesses in the main model.  This is the general idea being proposed by the research question.

Now given that there is training data for the combined model, i.e. the data misclassified or poorly regressed by the main data on the training data, the only missing piece is the architecture for the combined model.  It would be simple enough to concatenate text and image pixel intensities together as the combined model.  Yet, this seems like a poor option.  Existing research has shown that more abstract data types like images are poorly handled by NNs and better handled by CNNs.  It seems reasonable for a base combined model to somehow include CNNs when including any image data.  The main challenge is how to include image data with CNNs and text data.

As was explained in the related work, CNNs group features based on their location.  CNN layers can either compress the data in half or compare features across groups.  The ability for CNNs to behave like a normal NN by comparing values across groups is an advantage.  If text data can be represented as one group, and image data as another group, then these can be compared in a NN fashion within a CNN.

Part of the proposed methodology of this paper is to use a CNN for the boosting model.  This CNN will combine image and text data in a very particular way.  The CNN will be twice as wide as the image.  The left side will contain the image, the right side will contain the text with surrounding whitespace. By creating two squares in the CNN, when layers are halved in size, text and image data are technically kept separate within the CNN.  Each layer that compares data between groups will capture relationships between the text and image.  This allows for the CNN to evolve text and image data, while developing separate and combined features. The strength of this architecture is that it allows image data to undergo convolutions while interacting with the text data.  A weakness of the model is that text data is trained on a CNN.  Yet, as the related work showed, CNNs are able to obtain decent performance with text data.

The interesting part of this research is how it considers single-data type features.  These features are normally a problem when identifying joint data type features.  Yet, when only test data is used where single-data type features perform poorly, i.e. where misclassifications or poorly predicted regressions occur, this leaves an opportunity for training joint type features.  Boosting has shown great performance at improving structured data models.  It seems like there is opportunity for boosting to be used with NNs.  Traditionally boosting has been done with decision trees.  Yet, NN and different from structured data.  A weaker NN can be trained on a small amount of data.  By providing both data types to the NN, it is more likely that the NN will learn from these features.  Boosting is also nice because it only considers models that improve performance.  The final group of boosted CNN models can be given equal weight in the ensemble of each individual model.

%\section{Concepts}
%The concepts presented in the research question explore how well each standard model performs when trained on multiple data types.  The outcome of the research will give better detail on how well each model handles multiple data types.  Moreover, the research will also include how well a model performs when feed features from each individual model.  The study will give baseline information to future researchers on how well each model handles being trained on combination of text and image data.  Future research can build on the models which perform best in both experimenting with new architectures and preprocessing the data.

\subsection{Open Research Problems in IT}

\section{Purpose Statement}

\subsection{Signpost that establishes the central intent for the study?}

The topic of combining models of different data types offers a lot of promise.  If researcher's can find better ways to model complex data, i.e. data that contains multiple data types, then researcher's and practitioners will have better tools for analyzing real-world data.  Most data in the world is complex, interconnected, and can be represented in many ways.  For example, medical diagnosing patients makes use of visual data, e.g. X-Rays and a set of symptons, which can be represented as text data.  Having ways to combine the analysis of text and image data can produce models that make better predictions on these datasets. % maybe not the best place for examples

The central intent for the research is creating a methodology to improve the analysis of multi-data type datasets.  The research question poses both a methodology and architecture for analyzing this data.  The intent of the research question is to improve current model performance on these datasets, while showing that joint data type features exist.



\subsection{Is there a body of knowledge in IT you would like to contribute to?}

% Contribute to how complex models are analyzed

% How to provide this interracting data to existing models

%When it works and how to detect these relations
The body of knowledge I would like to contribute to is a better analysis of complex data types, specifically analyzing data that contains image and text data.  The task of analyzing complex data types is difficult because the complex data types interact differently, depending on their context and how they are used.  Moreover, because text and speech are different, a single machine learning model or architecture will introduce different biases on each data type, or will perform worse on one data type than the other. Many decisions must be made before a researcher can begin any analysis on text and speech data. Current practice generally analyzes each data type separately and combine their results. It is also common to transform one data type into the other for ease of processing, such as transcribing a picture into text and descriptive sentences. For example, these data types can be mixed into a single architecture that is better at analyzing the combination of data.  

I think the interests I can make research contributions to include complex data type preprocessing and both model architecture and combination of complex data types.  The research would focus on ways to build models that work on a combination of image and text data.  Many of the existing tools like attention models and the flexibility of CNN and NN models should go a long way to making a more combined analysis possible.  One challenge is the contex where the complex data analysis occurs.  Hopefully a combined model analysis and architecture will be general enough to apply to many context.  With the need of a general solution is a specific place to begin working on complex machine learning models.  One booming industry is the advertisement industry, and most ads are a combination of image and text data.  This field seems well suited to begin generating models for working with complex data types.  

\subsection{Is there an area of practice you would like to improve?}




