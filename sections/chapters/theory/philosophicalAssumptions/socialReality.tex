\subsection{Social Reality}

Multimodel research accepts that data types can be combined and encoded into new representations of that data.  These models can represent both data types and aid to knowledge gleamed from the dataset.  The overarching theory behind the multi-model research proposes that when multiple data types describe a phenomenon, that those data types are related. The theory is that some of the data from each data type is correlated, since the data is extracted from the same event.  It is as if each data type is a vector of information.  When each data types' vector is combined it creates a final data point, a master data point of sorts, which describes the phenomenon.  By combining each data type a machine learning model can better predict the master data point, or the phenomenon as a whole.  For example, social media advertisement contains an image and a description.  The user's reaction to an advertisement is likely a factor of not only the image or the description, but a factor of both the image and its description.  The theory poses that when data naturally occur together, each type of data contains correlations with the other type of data.

The research question also assumes that new machine learning architectures can better learn when designed to take advantage of correlations that may exist between the different data types.  These different architectures can either add data to existing models or improve upon existing models that usually only operate on one type of data.  By considering how each data type might interact, and how the model can learn from this interaction, the model may be able to better learn and represent the data and its features.  
