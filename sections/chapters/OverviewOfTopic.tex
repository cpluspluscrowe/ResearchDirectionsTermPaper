
\chapter{Problem}
IT is experiencing a massive growth in data.  This data is large, diverse, and differently structured.  The massive amount of diverse and unstructured data provides a problem in both processing and understanding this data.  This data is complicated and might contain more than plain text data.  The data very well could contain text and images, all of which is highly unstructured. The ability to understand and process this data can provide new data insights for researchers within this growing field. While there exist tools that can process text data, or which process image data, there are a lack of effective methods for analyzing text and image data.  The field finds itself fast growing and with a multitude of data without a good way to extract useable knowledge from that data. 

\chapter{Overview of the Topic}
The existing methodology for analyzing multi-data type datasets is to analyze each data type separately (Tsai 2011).  Current practice has seen a massive amount of success analyzing individual data types.  It is assumed that analyzing each data type separately will perform well enough for the vast majority of use cases (Srivastava, 2012).  While the assumption is true, there are many cases where increased model performance can improve knowledge on important datasets.  Also, multi-data type datasets are fairly new.  It is likely too early for practitioners and researchers to conclude that single-data type analysis is sufficient for most use-cases.  The field of machine learning research is constantly improving, and this paper makes strides to add to the growing field's knowledge.

Existing research on multi-data type datasets have almost exclusively used autoencoders because of their ability to represent complex data (Srivastava, 2014).  Yet, these models have problems.  They tend to learn single data type features and, in the end, many perform worse than their single-data type model counterparts.  The lack of performance by autoencoders on multi-data type features provides opportunity for improving modeling multi-data type features.

Within this area are many types of datasets.  There are datasets with many types of data, yet, images and text data are very common types of data in datasets.  This paper applies the tools and knowledge of applied machine learning to datasets containing both text and image data.  Yet, before speaking about solving this problem, it is worth talking about general principles when modeling any dataset.  

Most deep learning models consist of nodes and layers (Schmidhuber 2014).  The node and layer architecture provide flexible methods for learning and memorizing new data.  Extra layers provide opportunity for models to continue extracting data.  The combination of nodes and layers allow for different machine learning architectures.  Recent research has expanded upon nodes and layers to create more advanced architectures.  Some of these architectures have been labeled as deep, denoting the many layers they contain (Lecun et al., 2015). Other advances connect nodes in new ways, like connecting nodes to previous layers as with Recurrent Neural Networks (RNNs).  Each architecture and configuration have different benefits and learns data differently.  Some of these architectures will be more deeply covered in subsequent sections.  

It is worth a quicker dive into the basic mechanics of machine learning models. Machine learning models take inputs.  These inputs are transformed by models into outputs.  There are many manipulations on the data while it passes through the models.  The outputs depend on the type of machine learning.  Two common tasks within machine learning are classification and regression.  Classification models tend to output a probability for each classification through the final layer.  How the input is manipulated depends on the type of machine learning model.  One common type of machine learning model is an Artificial Neural Network (ANNs), which apply simple functions to the inputs, such as a signoid, Relu, or tanh function.  The model is generally composed of a few layers, where each node is connected to each node in its subsequent layer.  The network's input weights, for each layer are adjusted based on the model's error.  The Convolutional Neural Network (CNNs) is similar.  The CNN is also composed of nodes.  Half of the CNN learning is the same as ANNs, i.e. nodes connect to one another and go through a simple learning function like Relu.  The difference with CNNs is how the layers are connected.  Instead of connecting all layers together, CNNs split the data up into regional chunks, e.g. the upper right array of data.  The region of datas are connected.  One intersting behavior from the CNN is that region sizes are halved and joined.  In this way, regional behavior is captured and joined together in the output.  Another major type of machine learning model is a current Neural Network (RNNs).  RNNs are very similar to ANNs, except their layers are connected differently.  The RNNs layers do not always connect to the next layer, i.e. the layer's output may serve as input to a previous/the same layer.  This provides a cycle of input output information within the neural network.  This allows each layer to have some input from future layers, which creates a sense of network memory, where previous data serves as input to the network's current behavior.  

Machine learning models are often applied in particular ways.  For example, CNNs are often applied to images because they learn regional data well.  NN are often applied to text data because they are simple and can memorize the word distributions for langauges.  RNNs are often applied to videos because they can remember data from past inputs.  As will be covered more below, each of these simple models is regularly applied to single data types.  Applying these models to multiple data types will be covered in the related work and later sections.  

There are many major threads under development in the arena of applied machine learning.  These pertain to machine learning with each type of data. Examples of popular data types are images and text. These data types are popular because they are commonly available.  Such data are available in public repositories of data or through application APIs, such as text and image data that is available from utilizing the Twitter API.  Each data type has multiple facets.  More interestingly, certain machine learning algorithms perform better and differently for particular data types.  Neural Networks (NNs) perform very well on simple text learning tasks, yet struggle to learn image data.  Convolutional Neural Networks (CNNs) may underperform with text-analysis, but have excelled at image-analysis.  Analysis with sequences tends to use Long-Term Short Memory nodes (LSTMs) because the model incorporates sequence memory.

While there are many techniques for analyzing different types of data, society and academia has seen an increase in the amount of available data.  Moreover, not all data is the same.  There are many types of text data.  A researcher can analyze text data from Twitter, which looks different than text data from a research paper, which also looks different from text data translated from audio data.  The different data types and variations within data types are a challenge for researchers.  The creation of theories about Twitter text data represents advances in our knowledge of social media.  Creating models that are better at understanding text data that is created from audible speech can further research's understanding of text sentence structure.  These examples only illustrate that there are many ways to approach each data type, and each data type provides many ways to explore and understand its data.

When providing new machine learning insights, whether it be theories or achitectures, those theories need to be validated.  It is very common for papers to cite benchmark models on publically available data as a reference point for model performance.  There are many such benchmarks within the field of applied machine learning.  Many of these benchmarks are the performance of well-tuned machine learning algorithms on a given set of data.  Though this paper will not create a model and apply it to a benchmark, it will cite potential benchmarks for new papers or models implementing the theories presented.


%\chapter{Paper Overview}
%The paper focuses on a methodology for training a combined model.  This methodology is a natural conclusion based on the related work.  The paper begins in the next section with the related work.  The related work will begin describing current practice when modeling single data types.  The related work will discuss why single data types tend to be treated independently.  The main thrust of this reason is that particular architectures tend to perform better at modeling each data type.  In light of data-type specific architectures, the related work can explain the lack of research and success at modeling multiple-type datasets.

%The paper will then begin explaining existing work on mutli-type datasets.  Most of this work concerns generative models, and so provide no methodology for classification or regression with multi-type datasets.  Yet, these related work provide an important lesson.  The works delineate that when training a generative model with multiple data types, that the models tend to only learn single data type features.  Such single data type learning occurs because the features are stronger.  Also, in cases where the output type is of one type, the feedback encourages learning features whose type matches the output.

%With this knowledge, it is worth exploring ways to train multi-type models, while preventing single-data type learning.  The next section of the related work will discuss LSTMs because of their use with multiple data types and their ability to learn data transformations.  Since LSTMs have been considered as a way to translate between data types, they are worth mentioning here.  Other methods for training models on multiple data types will also be considered, such as Convolutional Neural Networks (CNNs) with videos and methods to translate images and videos into text.

%Having finished the related work, the paper will reiterate the problem.  Given a problem that needs solving, the research will propose a general solution and discuss an overview of possible implications for stakeholders, i.e. the field and the general public.  The next section will discuss existing theories that gave rise to the ideas in this paper.  Finally, the paper will describe its methodology.  The last section of the methodology will describe how the research will be evaluated both qualitatively and quantitatively.  

