\section{Abstract}

Datasets containing multiple data types are becoming more common.  Yet, there are far fewer papers on analyzing these types of datasets.  Most research analyzes each data type separately.  This paper proposes that joint features exist between data types that can be both detected and extracted.  This paper provides existing research on multi-type datasets, research questions, and proposed methodologies for analyzing this data.  Those questions focus on measuring standard model performance on multi-data type datasets and how concatenating models might facilitate detecting joint data type features.

%Existing research has great success at machine learning on single data types.  Yet, researchers have had less success on datasets containing multiple data types.  There are few papers on modeling data consisting of both text and images. These datasets are becoming more common and more relevant to researchers and data scientists.   This paper proposes challenges, research questions, and methodologies for approaching these datasets.     Yet, there are few papers on modeling data consisting of multiple data types.  Most of the existing research concerns translating between data types or separately modeling each feature.  This paper theorizes that when modeling a single phenomenon with multiple data types, that some information is only captured by combining features from each data type.  The implication is that machine learning models on multiple data types can be improved by capturing relationships between each data types.  This term paper describes the theories and proposed methodologies for extracting relations between the data types in order to improve the final model.
