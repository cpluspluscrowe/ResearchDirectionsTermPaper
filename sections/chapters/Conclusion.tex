\chapter{Conclusion}

Current approaches to modeling multi-type datasets tend to treat each data type as independent.  Yet, it is unlikely that each data type is independent.  Rather, it is likely that correlations exist across data types.  The trouble researchers face is detecting and extracting joint data type features.  This paper proposes that boosting based techniques can be applied to multi-data type datasets to extract joint data type features.  These methods seem promising based on the success boosting has exhibited on small datasets at model weak points.  If such a method is performant, then it provides researchers a way to detect joint data type features.  Moreover, the methodology is simple and can be implemented on existing benchmark datasets.

The research should generate answers and more paths for future research questions.  Such paths include the performance of combined models alongside existing single-data type models, the abilty to create weak models with DNNs, and architectures for multi-data type models.  The outcomes of the research will aid the understanding of multi-data type datasets, what can be known, and how joint features might be extracted.  Hopefully the proposed research proves helpful to those furthering multi-data type models within the domain of machine learning.




