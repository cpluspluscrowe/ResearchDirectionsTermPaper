\subsection{Why is it important?}

Data with multiple data types is becoming more common.  The new data provides a new challenge for researchers for analyzing the data.  One important challenge is drawing data from each data type, e.g. feature correlations between data types.  These joint data type features are a new source of knowledge for data scientists and researchers.  New knowledge has the potential for better performing models and improvement in current practice.

Current practice contains many datasets with multiple types.  Of particular note are examples in the medical field.  Search Kaggle alone gave provided many medical datasets with text and image data types, like: skin cancer datasets, Pulmonary Chest X-Ray abnormalities, Chest X-Ray Images for Pneumonia, DDSM Mammography, Blood Cell Images, CT Medical Images, Findings and Measuring Lungs in CT Data, MRI and Alzheimers, X-ray Bone Shadow Suppression, and many more.  These represent a small sample of a large number of datasets with important information.  Such examples and the feasibility of garnishing more quality data from these datasets is a good justification for further research.  It is likely that new information that better connect text data like symptoms with images could discover breakthroughs in diagnosing patients.  The potential benefits are far reaching but seem realistic if joint data type features can be extracted and added to current machine learning models.


