\subsection{What is known about this topic?}

Little has been invested creating benchmarks on multi-data type datasets.  Rather, researchers have moved straight to using auto-encoders because of their ability to handle complex data (Mao, 2015).  While the use of auto-encoders and Restricted Boltzmann Machines (RBMs) have justification, these are limited in their application.  Auto-encoders and RBMs are generative models, and therefore do not provide classification or regression outputs.  This leaves a gap for analyzing multi-data type datasets for the purpose of regression and classification.

There is existing research on multi-data type datasets.  As previously mentioned, these analyze each data type separately.  The benefit of this existing research is that benchmarks exist for performance and training time.  This provides a reference point for the model methodology proposed by this paper to be compared against.  


