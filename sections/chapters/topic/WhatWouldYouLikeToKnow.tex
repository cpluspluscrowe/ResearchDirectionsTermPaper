\subsubsection{What would you like to know?}

It would be beneficial to demarcate examples where complex data types are likely related.  Other research would benefit from proposed methodologies for demonstrably extracting relationships between data types.  A large amount of this term paper concerns describing those methodologies and how they might further the research of models with multiple data types.  The paper will also include recommendations for architectures that might perform well at extracting relationships between data types.


% I am interested in proving that features between data types are not mutually exclusive

% 2. I want to propose a method for extracting relationships between the two models





%I would like to know if LSTMs are capable of creating internal representations of each data type, so that LSTM models from each model can be combined.  I am curious if attention models can be used with single hidden layers to specify which data relationships between text and image data are most important/significant.  I am also curious how combining both text and image data in a CNN affects the data learned in each layer.  It would also be interesting to train a model on one data type, and then fix the top layers are retrain lower layers on the other data type to see if this gives performance benefits.  Such a fixed top/trained bottom data set might gain accuracy from learning some of the information from the other data set.  It would also be interesting to iteratively build a combined model based on two individual models, such as building a third combined model using weights from each independent image-CNN and text-NN model.  
