\subsection{What would you like to know?}

Existing research provides useful benchmarks for multi-data types like Flickr 8k and 30k.  Benchmarks are useful for comparing model performance.  Multi-data type datasets provide further information for the research topic.  If this paper's methodology create ways to detect joint data type features, then the methodology can be applied to these benchmarks.

It would be interesting to observe how boosting techniques affect what joint data type models learn.  It is hoped that providing negative feedback to the combined model when individual type features will allow the model to learn joint data type features. It will also be intersting if the model learns nothing, which may imply that no joint data type features exist.  It is also possible that joint data type featurs are closely related to the main features of each data type, which may be missed because of the negative feedback.  No matter what happens, the observed behavior will provide valuable feedback on a combined model's ability to learn when individual type features are restricted.
