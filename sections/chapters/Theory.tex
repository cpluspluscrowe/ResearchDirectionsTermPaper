\chapter{Theory} % What we understand

Theory speaks to which accepted theories the topic relies on.  One such assumption is that having more knowledge about a dataset will create machine learning models which perform better on the same data.  The goal of extracting knowledge from a dataset for a machine learning model is often known as feature extraction.  Practitioners who perform better feature extraction create models that perform better.  This feature extraction consists of extracting information from the model.  Another aspect is the quality of those features, i.e. do the features predict data behavior.  The proposition of this paper is more knowledge, i.e. features can be extracted when considering text and image datasets together, rather than each separately.  These joint features provide new knowledge, therefore new features, which should result in knowledge gained and a potentially better performing final machine learning model.  

Another theory is that data types are not always mutually exclusive in their features, i.e. that correlations can exist between features in different data types.  This theory has been shown to be true in videos by decomposing data into visual and audio data.  Yet, this theory has less support between text and image datasets.  This research provides opportunity to affirm or prove false for datasets consisting of text and image data.  

\subsection{Philosophical Assumptions} % Ontology

\subsection{Social Reality}

\subsection{What we understand to be true}

Some of the required theories are generally accepted. It is known when multiple data types exist, machine learning models can learn by including both data types in its models.  The difference is that usually these data types are separately modeled or each data type is transformed into a set of features which may be combined.  

Current theories have also shown that different data types can be analyzed with the same architecture.  For example, a CNN can perform image analysis and NLP tasks.  Yet, CNNs will perform worse on and differently on NN than on CNNs.  Also, a NN can process text and image data, but NN poorly draw out image features.  Existing studies have also transcribed one data type to another, such as sentence descriptions of images.  Yet, sentence description is generally the goal, not a transformation of data for more analysis. 



