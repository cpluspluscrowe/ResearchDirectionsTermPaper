\subsection{Open Research Problems in IT}

\subsubsection{1. Methodology for Autoencoders with Multiple Data Types}
Autoencoders can be used to create generative models. These work by learning ways to represent the data.  In this way, they can be used to transform data.  The related work covered a few examples of using autoencoders to transform one data type into another.  Yet, these encoders are only generative, not exact.  The models learn general patterns.  This makes autoencoders useful to learning general patterns from input to output.  Yet, there is not best way to use autoencoders with multiple data types.  There are many ways to provide data to autoencoders and arrange the orders of input and output.  This provides a gap in the field of autoencoders with multiple data types for proposing ways to best learn patterns between different data types.


\subsubsection{2. Detecting Joint Data Type Features}

