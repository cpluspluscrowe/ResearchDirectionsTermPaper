\subsection{Open Research Problems in IT}

\subsubsection{1. Detecting Joint Data Type Features}
There are many datasets that contain multiple data types.  Yet, there is no litmus test for identifying that correlated features exist between data types.  The creation of a method for identifying the existence of joint data type features would aid researcher's when deciding how to approach and learn on the dataset.  The creation of a method to detect these between dataset features is an open question this paper can begin to address.

\subsubsection{2. Methodology for Regression and Classification with Multiple Data Types}
Auto-encoders can be used to create generative models. These work by learning ways to represent the data.  In this way, they can be used to transform data.  The related work covered a few examples of using auto-encoders to transform one data type into another.  Yet, these encoders are only generative, not exact.  The models learn general patterns.  This makes autoencoders useful to learning general patterns from input to output, but not useful for the more common classification or regression problems.  This creates a gap for ways to perform classification and regression on multi-data datasets.  This gap exists because these datasets are complicated, joint data type features have weaker feedback signals (Ngiam et al., YEAR), and there is no common methodology for even detecting the existence of these features within a dataset.  


