\subsection{Central Intent for the Study}

The topic of combining models of different data types offers a lot of promise.  If researcher's can find better ways to model complex data, i.e. data that contains multiple data types, then researcher's and practitioners will have better tools for analyzing real-world data.  Most data in the world is complex, interconnected, and can be represented in many ways.  For example, medical diagnosing patients makes use of visual data, e.g. X-Rays and a set of symptons, which can be represented as text data.  Google has published the `Open Images` dataset with 9 million URLs with labels and 6,000 categories, and it is likely that the images and labels are related. There are many similar datasets with images and labels, such as the 'Google-Landmarks Dataset'. NBC recently published over 200,000 Tweets concerning Russia's involvement with the U.S. elections (2018).  Tweets use images and text to convey their message, so it is likely a dataset with image and text correlating features.  Insurance companies also make use of images and text, such as traffic accident images and the accident report.  The text describes events in the image, and so provide context for the image.  The ability to connect text and image features could improve insurance reporting for accidents.  Having ways to combine the analysis of text and image data can produce models that make better predictions on these datasets. % maybe not the best place for examples

The central intent for the research is creating a methodology to improve the analysis of multi-data type datasets.  The research question poses both a methodology and architecture for analyzing this data.  The intent of the research question is to improve current model performance on these datasets, while showing that joint data type features exist.


