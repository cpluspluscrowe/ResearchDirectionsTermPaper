\subsection{Is there a body of knowledge in IT you would like to contribute to?}

% Contribute to how complex models are analyzed

% How to provide this interracting data to existing models

%When it works and how to detect these relations
The body of knowledge I would like to contribute to is a better analysis of complex data types, specifically analyzing data that contains image and text data.  The task of analyzing complex data types is difficult because the complex data types interact differently, depending on their context and how they are used.  Moreover, because text and speech are different, a single machine learning model or architecture will introduce different biases on each data type, or will perform worse on one data type than the other. Many decisions must be made before a researcher can begin any analysis on text and speech data. Current practice generally analyzes each data type separately and combine their results. It is also common to transform one data type into the other for ease of processing, such as transcribing a picture into text and descriptive sentences. For example, these data types can be mixed into a single architecture that is better at analyzing the combination of data.  

I think the interests I can make research contributions to include complex data type preprocessing and both model architecture and combination of complex data types.  The research would focus on ways to build models that work on a combination of image and text data.  Many of the existing tools like attention models and the flexibility of CNN and NN models should go a long way to making a more combined analysis possible.  One challenge is the contex where the complex data analysis occurs.  Hopefully a combined model analysis and architecture will be general enough to apply to many context.  With the need of a general solution is a specific place to begin working on complex machine learning models.  One booming industry is the advertisement industry, and most ads are a combination of image and text data.  This field seems well suited to begin generating models for working with complex data types.  
