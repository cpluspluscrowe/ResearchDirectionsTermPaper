\subsection{What is the gap between what is known and what needs to be known?}

% gap is discovering and extracting ML features that only exist when both data types are considered

Research has analyzed data with single data types with a great degree of accuracy. There are commonplace techniques for analyzing each data type.  Moreover, these models are well-known and simple, such as CNNs for videos, NN for text, and RNNs for images. Given that simple models work well on simple data types, an interesting question is how well simple models process multiple data types.  This is a valid question and lies more in the arena of what needs to be known.

There has been less research analyzing multiple data types.  The research that has examined related data types focuses on the relationship between audio, movement, and images in videos. Of the research performing machine learning on multi-type datasets, these papers use autoencoders and LSTMs.  Yet, the models have tended to only learn single type features, since single type features are more significant or receive stronger negative feedback than joint data type features.  This leaves a major gap in modeling joint data type datasets.  There is a need for a methodology for both detecting and extracting joint data type features.

%A large gap is how well simple models perform with image and text data.  Research has also uncovered many advanced techniques for handling and analyzing complex data. Some of the complex techniques are worth mentioning when they interact with multiple data types. % I need more on what needs to be known



%One such method is to extract features from each individual model and combine these into a final model.  Another technique uses LSTMs or auto-encoders to translate from one data type to another.  The encoded data type can be combined with the second data type.  % Expand and give more examples based on RW



