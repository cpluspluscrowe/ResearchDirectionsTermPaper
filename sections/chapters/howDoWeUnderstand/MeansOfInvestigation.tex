\section{Means of Investigation} % (Methodology)

The main thrust of the research question is investigative.  The questions seek to benchmark how well simple architectures handle datasets with multiple data types. The overall goal is to identify if joint data type feature correlations exist and can be measured.  The research questions then ask if existing best-performing models can be combined near the end in order to detect these joint data type features.  The final research questions build on existing joint feature research and ask how well autoencoders and LSTMs perform at detecting joint data type features when compared against their single data type counterpart.

Testing the research questions is somewhat straightforward.  Standard architectures and trained models exist for each of the listed models, specifically NN, CNNs, LSTMs, and autoencoders. These models can be trained on each individual type of data.  A third model can be trained on both the image and text data.  The performance of each of the three models can be evaluated on a validation dataset.  The performance of the models can be recorded and compared.  There are any number of datasets that can be used as benchmarks.  Kaggle has Wine Reviews, Open URLs with website images and labels, Olympic athelete photos and results, CelebFaces dataset with images and attributes, Austin Animal Shelter Intakes and Outcomes, Skin Cancer with images and numerical attributes, 5,000 \#JustDoIt Tweets, Labeled images of meals, CAT and CT Scans with metadata, and many more datasets.  

\subsubsection{Qualitative/Inductive}

\subsubsection{Quantitative/Deductive}


