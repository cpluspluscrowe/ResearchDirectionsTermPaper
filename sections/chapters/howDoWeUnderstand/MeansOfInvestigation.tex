\section{Means of Investigation} % (Methodology)

The main thrust of the research question is investigative.  The questions seek to benchmark how well simple architectures handle datasets with multiple data types. The overall goal is to identify if joint data type feature correlations exist and can be measured.  The research questions then ask if existing best-performing models can be combined near the end in order to detect these joint data type features.  The final research questions build on existing joint feature research and ask how well autoencoders and LSTMs perform at detecting joint data type features when compared against their single data type counterpart.

Testing the research questions is somewhat straightforward.  Standard architectures and trained models exist for each of the listed models, specifically NN, CNNs, LSTMs, and autoencoders. These models can be trained on each individual type of data.  A third model can be trained on both the image and text data.  The performance of each of the three models can be evaluated on a validation dataset.  The performance of the models can be recorded and compared.  There are any number of datasets that can be used as benchmarks.  Kaggle has Wine Reviews, Open URLs with website images and labels, Olympic athelete photos and results, CelebFaces dataset with images and attributes, Austin Animal Shelter Intakes and Outcomes, Skin Cancer with images and numerical attributes, 5,000 \#JustDoIt Tweets, Labeled images of meals, CAT and CT Scans with metadata, and many more datasets.  

\subsubsection{Qualitative/Inductive}

The combined CNN model helps generate new insights and research questions due to its ability to visualize the activations at each layer.  A benefit of CNNs is the ability to visually depict features in each layer.  It is common for object classification to see the many layers of CNN and the activation in each layer.  One layer might find eyes, ears, tails, and other image features.  One benefit of using a combined CNN is that each layer's activations can be visualized and may provide insight and future research questions into what is occurring within a combined CNN.  

Boosting is a very popular idea that has yet to be applied to DNNs.  The performance of a boosting algorithm to DNNs has the potential to generate many related research questions.  For one, is it possible to create a weak model with boosting out of CNNs.  Secondly, how long does it take to generate a weak CNN?  A more interesting question is how many weak CNNs a boosting approach will create.  Futhermore, the performance of the ensemble of CNNs will be interesting, as well as how the ensemble of individual models and the CNN are combined.

The architecture of the CNN has the potential to generate more research.  The architecture seems like an intuitive way to combine text and image data.  It also opens up a general approach to combining image and text data.  Related research could explore the best configuration for dense and pooling layers with multi-data type architectures, since it is the dense layers that combine image and text data.  There is also the potential to create multi-type models with LSTMs and NN, then to compare their performance in boosting with the CNN.  The research direction has a lot of potential outcomes and may generate a lot of related research.


\subsubsection{Quantitative/Deductive}


The means of investigation are creating models that have promise for processing both image and text data.  These models can be trained and evaluated against existing benchmark datasets.  The performance of these models validates if the general theories work.  The deductive step assumes if the proposed architectures and methods perform well on existing benchmarks, then they will perform well on other data.  



