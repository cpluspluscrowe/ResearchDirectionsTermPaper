\section{Means of Investigation} % (Methodology)

Benchmark datasets already exist for multi-data type datasets.  Many of these datasets and performance metrics are on Kaggle and mentioned in the related work.  The benchmarks are useful because they provide the data for the experiment.  The benchmarks are also reference points for model performance and training time.  Moreover, there are many types of benchmarks containing text and image data.  This allows for the methodology to be tested for regression and classification performance.  This simplifies testing the methodology for multi-type datasets.

There are more nuanced aspects of the research question.  One such nuance is the assertion that the model will perform boosting on the dataset.  Boosting is a common machine learning technique and creates weak models that perform slightly better than random chance (Schapire et al., 2002).  One means of investigation will use existing boosting libraries to create an ensemble of boost models from the CNN architecture.  The investigation will be if the boosting library is able to generate weak models from the CNNs, implying that it is possible to create weak models from NNs.  

Building on the combined model, another means of investigation will be the performance of the final ensemble of models.  These models will contain a trained NN text-based model, CNN image-based model, and the ensemble of combined CNN models.  One means of investigation will be the classification accuracy/regression performance of averaging ball three models on a dataset.


\subsubsection{Qualitative/Inductive}

The combined CNN model helps generate new insights and research questions due to its ability to visualize the activations at each layer.  A benefit of CNNs is the ability to visually depict features in each layer.  It is common for object classification to see the many layers of CNN and the activation in each layer.  One layer might find eyes, ears, tails, and other image features.  One benefit of using a combined CNN is that each layer's activations can be visualized and may provide insight and future research questions into what is occurring within a combined CNN.  

Boosting is a very popular idea that has yet to be applied to DNNs.  The performance of a boosting algorithm to DNNs has the potential to generate many related research questions.  For one, is it possible to create a weak model with boosting out of CNNs.  Secondly, how long does it take to generate a weak CNN?  A more interesting question is how many weak CNNs a boosting approach will create.  Futhermore, the performance of the ensemble of CNNs will be interesting, as well as how the ensemble of individual models and the CNN are combined.

The architecture of the CNN has the potential to generate more research.  The architecture seems like an intuitive way to combine text and image data.  It also opens up a general approach to combining image and text data.  Related research could explore the best configuration for dense and pooling layers with multi-data type architectures, since it is the dense layers that combine image and text data.  There is also the potential to create multi-type models with LSTMs and NN, then to compare their performance in boosting with the CNN.  The research direction has a lot of potential outcomes and may generate a lot of related research.


\subsubsection{Quantitative/Deductive}


The means of investigation are creating models that have promise for processing both image and text data.  These models can be trained and evaluated against existing benchmark datasets.  The performance of these models validates if the general theories work.  The deductive step assumes if the proposed architectures and methods perform well on existing benchmarks, then they will perform well on other data.  



