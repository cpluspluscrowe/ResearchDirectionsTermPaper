\subsubsection{Quantitative/Deductive}

The deductive approach for this paper tests the theory that joint features exist.  The paper proposes a methodology that emphasizes and provides a environment that should foster the detection of joint data type features.  The deductive hypothesis is that joint features exist, and this hypothesis is tested in the proposed methodology.  The ability for boosting methods to create weak models would affirm that joint data type features likely exist, as single-data type features would have been detected by either single-data type models.  The very reason that boosting is not used for DNNs is because single type models are capable of analyzing, memorizing, and performing with huge feature sets, which is the weakpoint random forests have that boosting helps mitigate.

Another hypothesis is that a combined CNN is capable of detecting joint data type features.  This is closely connected to the previous hypothesis.  The ability to create weak models from boosting affirms this hypothesis, while an inability to generate weak models from boosting implies either that CNNs are unable to capture joint data type features, or that those features do not exist.  The ability to improve existing single data type models with a combined model is a third hypothesis tested by the research methodology.  If the final ensemble model, comprised of boosted CNNs, a text-based NN and image-based CNN outperforms current best-benchmarks, then the research has demonstrated that utilizing combined models can improve model performance.

