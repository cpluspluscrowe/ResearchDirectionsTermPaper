\subsubsection{Qualitative/Inductive}

The combined CNN model helps generate new insights and research questions due to its ability to visualize the activations at each layer.  A benefit of CNNs is the ability to visually depict features in each layer.  It is common for object classification to see the many layers of CNN and the activation in each layer.  One layer might find eyes, ears, tails, and other image features.  One benefit of using a combined CNN is that each layer's activations can be visualized and may provide insight and future research questions into what is occurring within a combined CNN.  

Boosting is a very popular idea that has yet to be applied to DNNs.  The performance of a boosting algorithm to DNNs has the potential to generate many related research questions.  For one, is it possible to create a weak model with boosting out of CNNs.  Secondly, how long does it take to generate a weak CNN?  A more interesting question is how many weak CNNs a boosting approach will create.  Futhermore, the performance of the ensemble of CNNs will be interesting, as well as how the ensemble of individual models and the CNN are combined.

The architecture of the CNN has the potential to generate more research.  The architecture seems like an intuitive way to combine text and image data.  It also opens up a general approach to combining image and text data.  Related research could explore the best configuration for dense and pooling layers with multi-data type architectures, since it is the dense layers that combine image and text data.  There is also the potential to create multi-type models with LSTMs and NN, then to compare their performance in boosting with the CNN.  The research direction has a lot of potential outcomes and may generate a lot of related research.

