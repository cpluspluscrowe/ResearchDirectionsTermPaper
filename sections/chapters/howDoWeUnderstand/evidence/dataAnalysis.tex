\subsubsection{Data Analysis}

After data and model collection, the researcher will have the necessary data and single-type models to begin the analysis.  Many of existing models are Keras or TensorFlow models, and so can be imported into existing projects.  These models are often pre-trained, or include a script to run which trains the models in a few hours.  From there, the third CNN architecture can be created.  As mentioned earlier, the combined CNN model has dimensions image-height x 2 * image-width.  The left-side of the CNN takes an image as input, and the right side takes the word vector padded with whitespace.  The following layers in the CNN have dense and pooling layers.  This CNN model can be passed to standard boosting libraries and provided as config during Keras training.  The boosting library will generate weak models.  The user can ensemble each individual model with the combined model.  This final model needs to testing, as each individual models are already trained and the weak models have been created.  The performance can be evaluated on the validation dataset.  The performance of the ensemble models can be reported on the benchmark data and recorded, as well as the number of weak models created by the boosting library.  The layers of the weak CNN models can also be visualized and included in the final results.  Each of these outputs will guide discussion on model performance, the ability to generate weak DNNs, and the ability to extract joint features from a combined model.

